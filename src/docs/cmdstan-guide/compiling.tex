\chapter{Compiling \Stan Programs with  \CmdStan}\label{compiling.chapter}

\noindent
Preparing a \Stan program to be run involves two steps,
%
\begin{enumerate}
\item translating the \Stan program to \Cpp, and
\item compiling the resulting \Cpp to an executable.
\end{enumerate}
%

\section{Translating and Compiling through {\tt\bfseries make}}\label{make-models.section}

The simplest way to compile a \Stan program is through the \code{make}
build tool, which runs in conjunction with a file called a \emph{makefile}.
The makefile contains rules to translate and compile a \Stan program,
as well as rules for building \CmdStan itself.
The arguments to the \code{make} command are called targets.
Given the name of a target, \code{make} looks for rules which
specify how to build that target.

The \code{make -f} flag specifies the name of the makefile.
If no \code{-f} flag is specified, make looks for a file named \code{makefile}
in the current working directory.
The \CmdStan home directory contains a makefile named \code{makefile}
which contains a target that 
combines the rules for translation and compiling a \Stan program,
so that we can build the \CmdStan executable via a single call to \code{make}.

Given a \Stan program called \code{my\_model.stan},
we tell \code{make} to build the executable \code{my\_model}
as follows, where \code{\$CMDSTAN\_HOME} is the path to the
\CmdStan home directory.
%
\begin{quote}
\begin{Verbatim}[fontshape=sl]
> make -f $CMDSTAN_HOME/makefile my\_model
\end{Verbatim}
\end{quote}
%$

\subsection{Translating and Compiling Test Models}

Open source Stan models, together with data simulators and real data,
are available from the Stan GitHub repository 
\url{https://github.com/stan-dev/example-models},
which contains basic examples used in the manual
as well as Stan versions of models found in books, including
Gelman and Hill 2007, Lee and Wagenmakers 2014, and most of the BUGS examples.

The directory \code{example-models/basic\_estimators}
contains the \Stan program \code{bernoulli.stan}
together with a data file in Rdump format \code{bernoulli.data.R}.
To build the \CmdStan executable model, working from this directory,
the following call to \code{make} suffices:
%
\begin{quote}
\begin{Verbatim}[fontshape=sl]
> make -f $CMDSTAN_HOME/makefile bernoulli
\end{Verbatim}
\end{quote}
%$
The following call will build an executable form of the Bernoulli
estimator. On Windows, replace \code{bernoulli} with \code{bernoulli.exe}.
This will translate the model \code{bernoulli.stan} to a \Cpp file and
compile a \CmdStan program using the generated \Cpp file, putting the
executable in
\code{example-models/basic\_estimators/bernoulli(.exe)}.

\subsection{Dependencies in {\tt\bfseries make}}

When executing a \code{make} target, all its dependencies are checked
to see if they are up to date, and if they are not, they are rebuilt.
If the \code{make} target to build the Bernoulli estimator is invoked
a second time, it will see that it is up to date, and will not
recompile the program.

If the file containing the \Stan program is updated, the next call to
\code{make} will rebuild the \CmdStan executable.



\subsection{Getting Help from the {\tt makefile}}

\CmdStan's \code{makefile}, which contains the top-level instructions to
\code{make}, provides help for targets and options.
%
\begin{quote}
\begin{Verbatim}[fontshape=sl]
> make -f $CMDSTAN_HOME/makefile help
\end{Verbatim}
\end{quote}
%$
From the \CmdStan home directory, it is not necessary to specify the makefile.
Furthermore, the help target is the default target, so the following two calls
to make are equivalent:
%
\begin{quote}
\begin{Verbatim}[fontshape=sl]
> make help
> make
\end{Verbatim}
\end{quote}
%



\subsection{Options to \code{make}}

\CmdStan allows users to change compilers, library versions for Eigen
and Boost, as well as compilation options such as optimization.

For a full list of options, see \refappendix{make-options}



\subsubsection{Clean Targets}

A very useful target is \code{clean-all}, invoked as
%
\begin{quote}
\begin{Verbatim}[fontshape=sl]
> make clean-all
\end{Verbatim}
\end{quote}
%
This removes the \CmdStan tools. This step is necessary when changing
compilers or other \code{make} options.
