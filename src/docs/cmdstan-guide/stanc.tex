\chapter{{\tt\bfseries stanc}: Translating \Stan to C++}\label{stanc.chapter}

\section{Building the \stanc Compiler}

Before the \stanc compiler can be used, it must be built. It can be
compiled directly using the makefile as follows. For Mac and Linux:
%
\begin{quote}
\begin{Verbatim}[fontshape=sl]
> make bin/stanc
\end{Verbatim}
\end{quote}
%
For Windows:
%
\begin{quote}
\begin{Verbatim}[fontshape=sl]
> make bin/stanc.exe
\end{Verbatim}
\end{quote}
%
To change the default compiler or the optimization level, see
\refappendix{make-options}.

\section{The \stanc Compiler}

The \stanc compiler converts \Stan programs to \Cpp concepts. The
first stage of compilation involves parsing the text of the \Stan
program.  If the parser is successful, the second stage of compilation
generates \Cpp code.  If the parser fails, it will provide a
diagnostic error message indicating the location in the input where
the failure occurred and reason for the failure.

The following example illustrates a fully qualified call to \stanc
to build the simple Bernoulli model. 

For Linux and Mac:
%
\begin{quote}
\begin{Verbatim}[fontshape=sl]
> cd <cmdstan-home>
> bin/stanc --name=bernoulli --o=bernoulli.hpp \
  stan/example-models/basic_estimators/bernoulli.stan 
\end{Verbatim}
\end{quote}
%
The backslash (\Verb|\|) indicates a continuation of the same line.

For Windows:
%
\begin{quote}
\begin{Verbatim}[fontshape=sl]
> cd <cmdstan-home>
> bin\stanc.exe --name=bernoulli --o=bernoulli.hpp ^
    stan/example-models/basic_estimators/bernoulli.stan 
\end{Verbatim}
\end{quote}
%
The caret (\Verb|^|) indicates continuation on Windows.

This call specifies the name of the model, here {\tt bernoulli}.
This will determine the name of the class implementing the model in
the \Cpp code.  Because this name is the name of a \Cpp class, it must
start with an alphabetic character (\code{a--z} or \code{A--Z}) and
contain only alphanumeric characters (\code{a--z}, \code{A--Z}, and
\code{0--9}) and underscores (\code{\_}) and should not conflict with
any \Cpp reserved keyword.  

The \Cpp code implementing the class is written to the file
\code{bernoulli.hpp} in the current directory.  The final argument,
\code{bernoulli.stan}, is the file from which to read the \Stan
program.

\section{Command-Line Options for {\tt\bfseries stanc}}

The model translation program \code{stanc} is called as follows.
%
\begin{quote}
\code{> stanc [options] {\slshape model\_file}}
\end{quote}
%
The argument \code{\slshape model\_file} is a path to a \Stan model
file ending in suffix \code{.stan}.  The options are as follows.
%
\begin{description}
%
\item[\tt {-}-help] 
\mbox{ } \\ 
Displays the manual page for \stanc.  If this option is selected,
nothing else is done.
%
\item[\tt {-}-version]
\mbox{ } \\ 
Prints the version of \stanc.  This is useful for bug reporting
and asking for help on the mailing lists.
%
\item[\tt {-}-name={\slshape class\_name}]
\mbox{ } \\ 
Specify the name of the class used for the implementation of the
\Stan model in the generated \Cpp code.  
\\[2pt]
Default: {\tt {\slshape class\_name = model\_file}\_model}
%
\item[\tt {-}-o={\slshape cpp\_file\_name}]
\mbox{ } \\ 
Specify the name of the file into which the generated \Cpp is written.
\\[2pt]
Default: {\tt {\slshape cpp\_file\_name} = {\slshape class\_name}.hpp}
%
\end{description}

